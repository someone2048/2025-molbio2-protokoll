\documentclass[11pt]{article}
\usepackage[a4paper, margin=2cm]{geometry}
\usepackage{xparse}
\usepackage[version=4]{mhchem}
\usepackage{siunitx}
\usepackage{graphicx}
\usepackage{titlesec}
\usepackage{xspace}
\usepackage{array}
\usepackage{xcolor,colortbl}
\usepackage{booktabs}
\usepackage{multirow}
\usepackage{pgfplots}
\usepackage{pgfplotstable}
\usepackage{amsmath}
\usepackage{xspace}
\usepackage{caption}
\usepackage{subcaption}
\usepackage{helvet}

\renewcommand{\thetable}{\Roman{table}}
\setcounter{secnumdepth}{4}

\NewDocumentCommand{\rruber}{}{\textit{E. coli}\xspace}
\NewDocumentCommand{\ecoli}{}{\textit{E. coli}\xspace}
\NewDocumentCommand{\ecolixloneblue}{}{\textit{E. coli} XL1 blue\xspace}
\NewDocumentCommand{\ecoliblgold}{}{\textit{E. coli} BL21 gold\xspace}


\title{Protokoll MOL601 Molekularbiologische Übungen 2 \\
Sommersemester 2025}
\author{Gruppe 2}
\date{24.03.2024}
    
\begin{document}

\maketitle

\section{Klonierung eines Esterase-Gens aus Rhodococcus ruber in einen Expressionsvektor in \ecoli}
\subsection{Durchführung}
\subsection{Ergebnisse \& Disskussion}


\section{Herstellen neuer \ecoli Stämme mit Antibiotikaresistenz und Fluoreszenz mittels P1 Transduktion}
\subsection{Durchführung}
\subsection{Ergebnisse \& Disskussion}


\section{TEMPLATES}
%% TABLE TEMPLATE
\begin{table}[h!]
    \centering
    \caption{\textbf{
    % TITLE HERE
    }}
    \begin{tabular}{lcc}
    \toprule
    \textbf{Coffein} & \textbf{Aspirin} & \textbf{Paracetamol} \\
    \midrule
    \textbf{Laufmittelfront (mm)} & 55  & 55  \\
    \textbf{Distanz (mm)}         & 7   & 21  \\
    \midrule
    \textbf{Rf}                   & 0.1 & 0.4 \\
    \bottomrule
    \end{tabular}
    \caption*{\small
        % SUB CAPTION HERE
    }
    \label{tab:template}
\end{table}

%% IMAGE TEMPLATE
\begin{figure}[h!]
    \centering
    \includegraphics{example-image}
    \caption{\small
     % IMAGE CAPTION
    }
    \label{fig:template}
\end{figure}


% bibliography
\begin{thebibliography}{9}

    \bibitem{skript}
    TU Graz, MOL601 Molekularbiologische Übungen 2 Skriptum, Sommersemester 2025

\end{thebibliography}




\end{document}